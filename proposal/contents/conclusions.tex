% !TeX encoding = UTF-8
% !TeX root = ../main.tex

%% ------------------------------------------------------------------------
%% Copyright (C) 2021 SJTUG
%% 
%% SJTUBeamer Example Document by SJTUG
%% 
%% SJTUBeamer Example Document is licensed under a
%% Creative Commons Attribution-NonCommercial-ShareAlike 4.0 International License.
%% 
%% You should have received a copy of the license along with this
%% work. If not, see <http://creativecommons.org/licenses/by-nc-sa/4.0/>.
%% -----------------------------------------------------------------------

\section{Conclusions}

\subsection{Benefits...}

\begin{frame}{Results}
	A consortium blockchain-based incentive model for crowdsensing system is
proposed
  \begin{itemize}
  \item \textbf{Benefits of consortium blockchain technology:} 
  	\begin{itemize}
  		\item resistant to the single point of failure (system security)
  		\item cooperative management (by requesters) reduces cost and enhances the flexibility of the system (selection criteria)
  	\end{itemize}
  \item \textbf{Benefits of hybrid incentive mechanism:}
  	\begin{itemize}
  		\item encourages workers to contribute valuable data (and penalizes malicious ones)
  		\item ensures favorables short-term and long-term incentives for workers
  	\end{itemize}
  \end{itemize}
\end{frame}


\subsection{... and limitations}

\begin{frame}{Limitations}
	Further research:
  \begin{enumerate}
  \item Dynamic situation where evaluations attributes are changing
  \item Optimization of consensus protocol (better performance)
  \item Further protection of worker privacy
  \end{enumerate}
  \begin{block}{Possible solutions}
  Application of ML techniques to blockchain-based system
  \end{block}
\end{frame}